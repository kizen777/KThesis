% Options for packages loaded elsewhere
\PassOptionsToPackage{unicode}{hyperref}
\PassOptionsToPackage{hyphens}{url}
\PassOptionsToPackage{dvipsnames,svgnames,x11names}{xcolor}
%
\documentclass[
  letterpaper,
  11pt,
  english,
  singlespacing,
  headsepline,
  japanese,
  oneside]{MastersDoctoralThesis}

\usepackage{amsmath,amssymb}
\usepackage{iftex}
\ifPDFTeX
  \usepackage[T1]{fontenc}
  \usepackage[utf8]{inputenc}
  \usepackage{textcomp} % provide euro and other symbols
\else % if luatex or xetex
  \usepackage{unicode-math}
  \defaultfontfeatures{Scale=MatchLowercase}
  \defaultfontfeatures[\rmfamily]{Ligatures=TeX,Scale=1}
\fi
\usepackage{lmodern}
\ifPDFTeX\else  
    % xetex/luatex font selection
\fi
% Use upquote if available, for straight quotes in verbatim environments
\IfFileExists{upquote.sty}{\usepackage{upquote}}{}
\IfFileExists{microtype.sty}{% use microtype if available
  \usepackage[]{microtype}
  \UseMicrotypeSet[protrusion]{basicmath} % disable protrusion for tt fonts
}{}
\makeatletter
\@ifundefined{KOMAClassName}{% if non-KOMA class
  \IfFileExists{parskip.sty}{%
    \usepackage{parskip}
  }{% else
    \setlength{\parindent}{0pt}
    \setlength{\parskip}{6pt plus 2pt minus 1pt}}
}{% if KOMA class
  \KOMAoptions{parskip=half}}
\makeatother
\usepackage{xcolor}
\usepackage[paper=a4paper,inner=2.5cm,outer=3.8cm,bindingoffset=.5cm,top=1.5cm,bottom=1.5cm]{geometry}
\setlength{\emergencystretch}{3em} % prevent overfull lines
\setcounter{secnumdepth}{5}
% Make \paragraph and \subparagraph free-standing
\makeatletter
\ifx\paragraph\undefined\else
  \let\oldparagraph\paragraph
  \renewcommand{\paragraph}{
    \@ifstar
      \xxxParagraphStar
      \xxxParagraphNoStar
  }
  \newcommand{\xxxParagraphStar}[1]{\oldparagraph*{#1}\mbox{}}
  \newcommand{\xxxParagraphNoStar}[1]{\oldparagraph{#1}\mbox{}}
\fi
\ifx\subparagraph\undefined\else
  \let\oldsubparagraph\subparagraph
  \renewcommand{\subparagraph}{
    \@ifstar
      \xxxSubParagraphStar
      \xxxSubParagraphNoStar
  }
  \newcommand{\xxxSubParagraphStar}[1]{\oldsubparagraph*{#1}\mbox{}}
  \newcommand{\xxxSubParagraphNoStar}[1]{\oldsubparagraph{#1}\mbox{}}
\fi
\makeatother


\providecommand{\tightlist}{%
  \setlength{\itemsep}{0pt}\setlength{\parskip}{0pt}}\usepackage{longtable,booktabs,array}
\usepackage{calc} % for calculating minipage widths
% Correct order of tables after \paragraph or \subparagraph
\usepackage{etoolbox}
\makeatletter
\patchcmd\longtable{\par}{\if@noskipsec\mbox{}\fi\par}{}{}
\makeatother
% Allow footnotes in longtable head/foot
\IfFileExists{footnotehyper.sty}{\usepackage{footnotehyper}}{\usepackage{footnote}}
\makesavenoteenv{longtable}
\usepackage{graphicx}
\makeatletter
\newsavebox\pandoc@box
\newcommand*\pandocbounded[1]{% scales image to fit in text height/width
  \sbox\pandoc@box{#1}%
  \Gscale@div\@tempa{\textheight}{\dimexpr\ht\pandoc@box+\dp\pandoc@box\relax}%
  \Gscale@div\@tempb{\linewidth}{\wd\pandoc@box}%
  \ifdim\@tempb\p@<\@tempa\p@\let\@tempa\@tempb\fi% select the smaller of both
  \ifdim\@tempa\p@<\p@\scalebox{\@tempa}{\usebox\pandoc@box}%
  \else\usebox{\pandoc@box}%
  \fi%
}
% Set default figure placement to htbp
\def\fps@figure{htbp}
\makeatother

\usepackage[utf8]{inputenc} % Required for inputting international characters
%\usepackage[T1]{fontenc} % Output font encoding for international characters; causes problems for xelatex

%\usepackage{mathpazo} % Use the Palatino font by default

\usepackage[backend=biber, style=authoryear, natbib=true]{biblatex} % Use the bibtex backend with the authoryear citation style (which resembles APA)

\usepackage[autostyle=true]{csquotes} % Required to generate language-dependent quotes in the bibliography


%----------------------------------------------------------------------------------------
%	MARGINS
%----------------------------------------------------------------------------------------

\geometry{
	headheight=4ex,
	includehead,
	includefoot
}

\raggedbottom

\AtBeginDocument{
\hypersetup{pdftitle=\ttitle} % Set the PDF's title to your title
\hypersetup{pdfauthor=\authorname} % Set the PDF's author to your name
\hypersetup{pdfkeywords=\keywordnames} % Set the PDF's keywords to your keywords
}

\usepackage[utf8]{inputenc}

%----------------------------------------------------------------------------------------
%	SETTING JAPANESE FONTS
%----------------------------------------------------------------------------------------

% 日本語フォントの設定
\usepackage{xeCJK}
\setCJKmainfont{Hiragino Mincho ProN}
\setCJKsansfont{Hiragino Sans}
\setCJKmonofont{Hiragino Sans}

% 日本語用の行送りの設定
\XeTeXlinebreaklocale "ja"
\XeTeXlinebreakskip = 0pt plus 1pt

\usepackage[backend=biber, style=authoryear, natbib=true]{biblatex}
\makeatletter
\@ifpackageloaded{bookmark}{}{\usepackage{bookmark}}
\makeatother
\makeatletter
\@ifpackageloaded{caption}{}{\usepackage{caption}}
\AtBeginDocument{%
\ifdefined\contentsname
  \renewcommand*\contentsname{Table of contents}
\else
  \newcommand\contentsname{Table of contents}
\fi
\ifdefined\listfigurename
  \renewcommand*\listfigurename{List of Figures}
\else
  \newcommand\listfigurename{List of Figures}
\fi
\ifdefined\listtablename
  \renewcommand*\listtablename{List of Tables}
\else
  \newcommand\listtablename{List of Tables}
\fi
\ifdefined\figurename
  \renewcommand*\figurename{Figure}
\else
  \newcommand\figurename{Figure}
\fi
\ifdefined\tablename
  \renewcommand*\tablename{Table}
\else
  \newcommand\tablename{Table}
\fi
}
\@ifpackageloaded{float}{}{\usepackage{float}}
\floatstyle{ruled}
\@ifundefined{c@chapter}{\newfloat{codelisting}{h}{lop}}{\newfloat{codelisting}{h}{lop}[chapter]}
\floatname{codelisting}{Listing}
\newcommand*\listoflistings{\listof{codelisting}{List of Listings}}
\makeatother
\makeatletter
\makeatother
\makeatletter
\@ifpackageloaded{caption}{}{\usepackage{caption}}
\@ifpackageloaded{subcaption}{}{\usepackage{subcaption}}
\makeatother

\usepackage[]{biblatex}
\addbibresource{references.bib}
\usepackage{bookmark}

\IfFileExists{xurl.sty}{\usepackage{xurl}}{} % add URL line breaks if available
\urlstyle{same} % disable monospaced font for URLs
\hypersetup{
  pdftitle={中高齢者における自転車運動の実施が 心身の健康に及ぼす影響},
  pdfauthor={佐々木~毅然},
  colorlinks=true,
  linkcolor={blue},
  filecolor={Maroon},
  citecolor={magenta},
  urlcolor={red},
  pdfcreator={LaTeX via pandoc}}


\thesistitle{中高齢者における自転車運動の実施が\\
心身の健康に及ぼす影響} % Your thesis title, this is used in the title and abstract, print it elsewhere with \ttitle
\supervisor{内田~勇人~教授} % Your supervisor's name, this is used in the title page, print it elsewhere with \supname
\examiner{} % Your examiner's name, this is not currently used anywhere in the template, print it elsewhere with \examname
\degree{博士論文} % Your degree name, this is used in the title page and abstract, print it elsewhere with \degreename
\author{佐々木~毅然} % Your name, this is used in the title page and abstract, print it elsewhere with \authorname
\addresses{} % Your address, this is not currently used anywhere in the template, print it elsewhere with \addressname

\subject{} % Your subject area, this is not currently used anywhere in the template, print it elsewhere with \subjectname
\keywords{} % Keywords for your thesis, this is not currently used anywhere in the template, print it elsewhere with \keywordnames
\university{} % Your university's name, this is used in the title page and abstract, print it elsewhere with \univname
\department{兵庫県立大学大学院環境人間学研究科} % Your department's name, this is used in the title page and abstract, print it elsewhere with \deptname
\group{環境人間学専攻} % Your research group's name and URL, this is used in the title page, print it elsewhere with \groupname
\faculty{Applied Math
Group} % Your faculty's name and URL, this is used in the title page and abstract, print it elsewhere with \facname

\setcounter{tocdepth}{2} % The depth to which the document sections are printed to the table of contents
\begin{document}
\frontmatter % Use roman page numbering style (i, ii, iii, iv...) for the pre-content pages

\pagestyle{plain} % Default to the plain heading style until the thesis style is called for the body content

%----------------------------------------------------------------------------------------
%	TITLE PAGE
%----------------------------------------------------------------------------------------

\begin{titlepage}
\begin{center}

\vspace*{.06\textheight}
{\scshape\LARGE \univname\par}\vspace{1.5cm} % University name
\textsc{\LARGE 博士論文}\\[0.5cm] % Thesis type

\HRule \\[0.3cm] % Horizontal line
{\huge \bfseries \ttitle\par}\vspace{0.4cm} % Thesis title
\HRule \\[1.5cm] % Horizontal line

\vspace{0.1cm}  % 0.1cmの垂直空白
\includegraphics[height=3.5cm]{images/kendailogo.png} % University/department logo

%\begin{minipage}[t]{0.4\textwidth}
%\begin{flushleft} \large
%\emph{Author:}\\
%%\authorname
%%\end{flushleft}
%\end{minipage}

%\begin{minipage}[t]{0.4\textwidth}
%\begin{flushright} \large
%\emph{Supervisor:} \\
%%
%\supname
%
%\end{flushright}
%\end{minipage}\\[3cm]

\textit{}\\[2.0cm]
\LARGE 指導教員 \supname\\[0.5cm] % Supervisor name
 
\vfill

%\large \textit{A thesis submitted in fulfillment of the requirements\\ for the degree of \degreename}\\[0.3cm] % University requirement text
\textit{}\\[0.5cm]
\LARGE\deptname\\
\LARGE\groupname\\[0.3cm] % Research group name and department name
\LARGE\authorname\\[0cm]  % 著者名を追加し、0.5cmの空白を設定

\vfill




 
\vfill
\end{center}
\end{titlepage}

% 

\begingroup
\hypersetup{linkcolor=black}

\tableofcontents % Prints the main table of contents

\listoffigures % Prints the list of figures

\listoftables % Prints the list of tables

\endgroup



%----------------------------------------------------------------------------------------
%	DECLARATION PAGE
%----------------------------------------------------------------------------------------
\begin{declaration}
\addchaptertocentry{\authorshipname} % Add the declaration to the table of contents
\input{"Frontmatter/declaration.tex"}

\end{declaration}

\cleardoublepage

%----------------------------------------------------------------------------------------
%	QUOTATION PAGE
%----------------------------------------------------------------------------------------

\vspace*{0.2\textheight}

\noindent``{\itshape Thanks to my solid academic training, today I can
write hundreds of words on virtually any topic without possessing a
shred of information, which is how I got a good job in
journalism.}''\bigbreak

\hfill Dave Barry


%----------------------------------------------------------------------------------------
%	ABSTRACT PAGE
%----------------------------------------------------------------------------------------

\begin{abstract}
\addchaptertocentry{\abstractname} % Add the abstract to the table of contents
Lorem ipsum dolor sit amet, consectetur adipiscing elit. Aliquam quis
accumsan ante. Quisque lorem metus, varius id urna eget, lacinia dapibus
sem. Etiam laoreet, quam ac mollis congue, arcu leo dictum neque, nec
euismod sem enim luctus odio. Donec condimentum tortor sit amet mollis
volutpat. Donec ornare libero vel velit malesuada consectetur.
Vestibulum in sem non justo dignissim congue at quis erat. Integer quis
erat vitae mi maximus fringilla tristique nec odio. Morbi non ipsum
sapien. Vestibulum tortor est, ultricies in eros et, bibendum iaculis
justo. Cras pellentesque enim quam, id pretium lacus lacinia non.
Integer velit neque, ultrices a malesuada vel, imperdiet quis enim.
Quisque eu facilisis urna, ut faucibus lorem. Donec mollis turpis sed
arcu venenatis interdum. Nulla facilisis tortor ac scelerisque
consequat.
\end{abstract}


%----------------------------------------------------------------------------------------
%	ACKNOWLEDGEMENTS
%----------------------------------------------------------------------------------------

\begin{acknowledgements}
\addchaptertocentry{\acknowledgementname} % Add the acknowledgements to the table of contents
\input{"Frontmatter/acknowledgements.tex"}
\end{acknowledgements}



%----------------------------------------------------------------------------------------
%	ABBREVIATIONS
%----------------------------------------------------------------------------------------

\input{"Frontmatter/abbreviations.tex"}



%----------------------------------------------------------------------------------------
%	SYMBOLS
%----------------------------------------------------------------------------------------

\input{"Frontmatter/symbols.tex"}


%----------------------------------------------------------------------------------------
%	DEDICATION
%----------------------------------------------------------------------------------------

\dedicatory{\input{"Frontmatter/dedication.tex"}} 


%----------------------------------------------------------------------------------------
%	THESIS CONTENT - CHAPTERS
%----------------------------------------------------------------------------------------

\mainmatter % Begin numeric (1,2,3...) page numbering

\pagestyle{thesis} % Return the page headers back to the "thesis" style
% Define some commands to keep the formatting separated from the content 
\newcommand{\keyword}[1]{\textbf{#1}}
\newcommand{\tabhead}[1]{\textbf{#1}}
\newcommand{\code}[1]{\texttt{#1}}
\newcommand{\file}[1]{\texttt{\bfseries#1}}
\newcommand{\option}[1]{\texttt{\itshape#1}}


\bookmarksetup{startatroot}

\chapter{序章}\label{sec-Chapter1}

\bookmarksetup{startatroot}

\chapter{序章}\label{sec-Chapter1}

\bookmarksetup{startatroot}

\chapter{Chapter 2 Title}\label{sec-Chapter2}

\section{Main Section 1}\label{main-section-1}

Lorem ipsum dolor sit amet, consectetur adipiscing elit. Aliquam
ultricies lacinia euismod. Nam tempus risus in dolor rhoncus in interdum
enim tincidunt. Donec vel nunc neque. In condimentum ullamcorper quam
non consequat. Fusce sagittis tempor feugiat. Fusce magna erat, molestie
eu convallis ut, tempus sed arcu. Quisque molestie, ante a tincidunt
ullamcorper, sapien enim dignissim lacus, in semper nibh erat lobortis
purus. Integer dapibus ligula ac risus convallis pellentesque.

\begin{equation}\phantomsection\label{eq-pmb}{
\int_{A_i}{\lambda (\pmb{\mu})}
}\end{equation}

\begin{equation}\phantomsection\label{eq-mathbf}{
\int_{A_i}{\lambda (\mathbf{x})}
}\end{equation}

\subsection{Subsection 1}\label{subsection-1}

Nunc posuere quam at lectus tristique eu ultrices augue venenatis.
Vestibulum ante ipsum primis in faucibus orci luctus et ultrices posuere
cubilia Curae; Aliquam erat volutpat. Vivamus sodales tortor eget quam
adipiscing in vulputate ante ullamcorper. Sed eros ante, lacinia et
sollicitudin et, aliquam sit amet augue. In hac habitasse platea
dictumst.

\subsection{Subsection 2}\label{subsection-2}

Morbi rutrum odio eget arcu adipiscing sodales. Aenean et purus a est
pulvinar pellentesque. Cras in elit neque, quis varius elit. Phasellus
fringilla, nibh eu tempus venenatis, dolor elit posuere quam, quis
adipiscing urna leo nec orci. Sed nec nulla auctor odio aliquet
consequat. Ut nec nulla in ante ullamcorper aliquam at sed dolor.
Phasellus fermentum magna in augue gravida cursus. Cras sed pretium
lorem. Pellentesque eget ornare odio. Proin accumsan, massa viverra
cursus pharetra, ipsum nisi lobortis velit, a malesuada dolor lorem eu
neque.

\section{Main Section 2}\label{main-section-2}

Sed ullamcorper quam eu nisl interdum at interdum enim egestas. Aliquam
placerat justo sed lectus lobortis ut porta nisl porttitor. Vestibulum
mi dolor, lacinia molestie gravida at, tempus vitae ligula. Donec eget
quam sapien, in viverra eros. Donec pellentesque justo a massa fringilla
non vestibulum metus vestibulum. Vestibulum in orci quis felis tempor
lacinia. Vivamus ornare ultrices facilisis. Ut hendrerit volutpat
vulputate. Morbi condimentum venenatis augue, id porta ipsum vulputate
in. Curabitur luctus tempus justo. Vestibulum risus lectus, adipiscing
nec condimentum quis, condimentum nec nisl. Aliquam dictum sagittis
velit sed iaculis. Morbi tristique augue sit amet nulla pulvinar id
facilisis ligula mollis. Nam elit libero, tincidunt ut aliquam at,
molestie in quam. Aenean rhoncus vehicula hendrerit.

\bookmarksetup{startatroot}

\chapter*{References}\label{references}
\addcontentsline{toc}{chapter}{References}

\markboth{References}{References}

\printbibliography[heading=none]

\cleardoublepage
\phantomsection
\addcontentsline{toc}{part}{Appendices}
\appendix

\chapter{Frequently Asked Questions}\label{sec-appA}

\section{How do I change the colors of
links?}\label{how-do-i-change-the-colors-of-links}

Pass in \texttt{urlcolor:} in yaml. Or set these in the
include-in-header file.

\noindent If you want to completely hide the links, you can use:

\{\small\verb!\hypersetup{allcolors=.}!\}, or even better:

\{\small\verb!\hypersetup{hidelinks}!\}.

\noindent If you want to have obvious links in the PDF but not the
printed text, use:

\{\small\verb!\hypersetup{colorlinks=false}!\}.





\end{document}
